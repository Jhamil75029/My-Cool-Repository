\chapter{Git}
\section{¿Qué es Git?}
Git es es un sistema de control de versiones distribuido que permite a los desarrolladores llevar un registro de los cambios en sus archivos y coordinar el trabajo en proyectos de software de manera eficiente.\\

Distribuido quiere decir que no depende de un único sitio en específico para almacenar su código. Existen muchos controladores de versiones, pero no todos son distribuidos, eso significa que por ejemplo si el servidor donde tienen almacenado el código llega a reventar, entonces perderían todo el código.\\

En cambio, un sistema distribuido permite tener a todos los colaboradores una copia del proyecto en sus propios equipos, y en caso de que el servidor principal reventase, el proyecto seguiría existiendo en las máquinas personales de los desarrolladores y se podría recuperar.

\section{Comandos de Git}

\subsection{Git Init}
Para empezar a usar git, primero se necesita un repositorio local, es decir una carpeta, sobre la cual git va a ir controlando y guardando sus versiones y revisando sus cambios.\\

Una vez creada la carpeta, desde la terminal hay que posicionarse sobre esa carpeta (en bash con cd ~/directorio).\\

Una vez hecho esto en la terminal se va a escribir, \textbf{git init} y esto automáticamente creará una carpeta oculta llamada ".git" en la carpeta.\\
\begin{figure}[h!]
	\centering
	\includegraphics[scale = 0.7]{Images/gitinit}
	\caption{Git Init ejecutado en la terminal}
	
	\includegraphics[scale = 0.7]{Images/dotgit}
	\caption{Carpeta Oculta .git en el repositorio}
\end{figure}

\subsection{Git Add y Git Commit}
Ahora que git está monitoreando el repositorio podemos crear un fichero de lo que gustemos y git lo detectará, pero no lo guardará automáticamente, es importante entender los 3 estados en los que puede estar un fichero en un repositorio con git.\\
\begin{enumerate}
	\item \textbf{modified:} El archivo tiene cambios pero aún no fueron marcados para ser confirmados, se encuentra en el directorio de trabajo.
	\item \textbf{staged:} Son los archivos que fueron modificados y confirmados para guardarse en el repositorio local, se encuentran en un área temporal transitoria.
	\item \textbf{commited:} El archivo se encuentra grabado en el repositorio local.
\end{enumerate}

Cuando creamos un archivo nuevo, git no lo reconocerá automáticamente, aparecerá como "untracked" para que git empiece a hacerle seguimiento hay que ejecutar el comando: \textbf{git add nombre-del-archivo}, y así pasará al estado de \textbf{staged}.

Una vez que todos los archivos se encuentren en el estado staged, estaŕan listos para que se les pueda realizar un commit, básicamente les estamos sacando una foto en el momento actual a todos los archivos, guardando el momento. Esto se hace con el comando \textbf{git commit}, este comando nos abrirá un editor de código para colocar un nombre al commit, una vez colocado el nombre el commit ya queda realizado.


Git ammend
Git Restore